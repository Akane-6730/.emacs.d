% =========================================
% Basic Utilities and Formatting
% =========================================
% Inclusion of graphics
\usepackage{graphicx}
% Tables that span multiple pages
\usepackage{longtable}
% Professional tables
\usepackage{booktabs}
% Underlining text (normalem prevents overriding \emph)
\usepackage[normalem]{ulem}

% =========================================
% Page Layout and Headers
% =========================================
% Page margins and dimensions
\usepackage[margin=1in]{geometry}
% Headers and footers customization
\usepackage{fancyhdr}
\pagestyle{fancy}
\cfoot{\thepage}

% =========================================
% Typography and Fonts (Chinese Support)
% =========================================
% Chinese typesetting support
\usepackage{ctex}
\setmonofont{Consolas}
\setCJKmonofont{SimHei}

% =========================================
% Mathematics
% =========================================
% Standard AMS math packages
\usepackage{amsmath, amssymb}
% Extension packages for amsmath
\usepackage{mathtools}
% Bold math symbols
\usepackage{bm}

% =========================================
% Colors and Hyperlinks
% =========================================
% Color support
\usepackage{xcolor}
\definecolor{ZJUBlue}{RGB}{0,63,136}

% Hyperlinks and PDF metadata
\usepackage[colorlinks=true, allcolors=ZJUBlue]{hyperref}

% =========================================
% Miscellaneous
% =========================================
% Footnote customization (bottom ensures footnotes stay at page bottom)
\usepackage[bottom]{footmisc}

% =========================================
% Code Highlighting
% =========================================
% Syntax highlighting using Pygments
\usepackage{minted}
\usemintedstyle{solarized-light}

% =========================================
% Theorem Environments (Beautiful Boxes)
% =========================================
\usepackage{amsthm}
\usepackage[many]{tcolorbox}
\tcbuselibrary{theorems,breakable,skins}

% Base style for theorem boxes
\tcbset{
	lalustyle/.style={
			breakable,
			enhanced jigsaw,
			sharp corners,
			colframe=#1!50!black,      % Border: 50% base color + 50% black
			colback=white,              % Content background: white
			colbacktitle=#1!5,         % Title background: 5% base color (very light)
			coltitle=black,             % Title text: black
			fonttitle=\bfseries,        % Title font: bold
			% Border settings (no left/right borders)
			leftrule=0pt,
			rightrule=0pt,
			toprule=1pt,
			titlerule=1pt,
			bottomrule=1pt,
			% Title padding
			toptitle=1mm,
			bottomtitle=1mm,
			separator sign={\quad},
			% Dashed lines for page breaks
			overlay first={
					\draw[#1!50!black, line width=1pt, dashed]
					([yshift=-0.5pt]frame.south west) -- ([yshift=-0.5pt]frame.south east);
				},
			overlay middle={
					\draw[#1!50!black, line width=1pt, dashed]
					([yshift=0.5pt]frame.north west) -- ([yshift=0.5pt]frame.north east);
					\draw[#1!50!black, line width=1pt, dashed]
					([yshift=-0.5pt]frame.south west) -- ([yshift=-0.5pt]frame.south east);
				},
			overlay last={
					\draw[#1!50!black, line width=1pt, dashed]
					([yshift=0.5pt]frame.north west) -- ([yshift=0.5pt]frame.north east);
				},
		}
}

% Define theorem environments (Chinese)
\newtcbtheorem[number within=section]{definition}{定义}{lalustyle={red}}{def}
\newtcbtheorem[number within=section]{theorem}{定理}{lalustyle={violet}}{thm}
\newtcbtheorem[number within=section]{example}{例}{lalustyle={blue}}{ex}

% Proof/Solution box
\NewTColorBox{tcblaluthmbox}{mmm}{
	lalustyle={teal},
	colframe=teal!80!black,
	title=#1\IfBlankF{#2}{\quad#2},
	#3,
}

% Renew proof environment (amsthm already defines it, so we use RenewDocumentEnvironment)
\RenewDocumentEnvironment{proof}{O{}O{}}{
\begin{tcblaluthmbox}{{\bfseries 证明}}{#1}{#2}
	}{
	\hspace*{\fill}$\square$\end{tcblaluthmbox}
}
